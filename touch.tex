\documentclass[a4paper,12pt,twoside]{article}

\usepackage[a4paper, inner=2cm, outer=2cm,top=2cm, bottom=2cm]{geometry}
\usepackage[utf8]{inputenc}
\usepackage[portuges]{babel}
\usepackage{aeguill}
\usepackage{indentfirst}
\usepackage{xcolor}
\usepackage{graphicx}
\usepackage{titlesec}
\usepackage{titling}
\usepackage{graphicx}
\usepackage{wrapfig}
\usepackage{caption}
\usepackage{subcaption}
\usepackage{enumitem}
\usepackage{amsmath}

\usepackage{listings}
\usepackage{color}
\input{~/docs/latex/lstset.tex}

\usepackage[backend=biber]{biblatex}
\addbibresource{~/docs/latex/refs.bib}

\renewcommand{\maketitle}{
\begin{center}
{\Huge\bfseries
\thetitle}
\vspace{.25em}
\Large{09/07/2020 - 15/07/2020}
\end{center}
}

\title{Relatório de Iniciação Científica}
\author{Lucas Budde Mior}

\begin{document}

\newgeometry{inner=3cm, outer=2cm, top=3cm, bottom=2cm}
\maketitle
\section{Objetivos}

\begin{itemize}
    \item{Estudar capítulo 3 \cite{gonzales}, terminando a parte de histogramas e filtros. Aplicar técnicas de processamento usando Octave \cite{octave}}. 
\end{itemize}

\section{Resultados}
\subsection{Uso de estatísticas do histograma para aprimoramento de imagem}
Sendo \(p(r_i)\) o histograma normalizado de uma imagem, sabe-se que \(p(r_i)\) é uma estimativa da probabilidade que a intensidade \(r_i\) ocorre na imagem.
Se uma imagem tem L níveis de intensidade, o enésimo \textit{momento central} de \(r\) é definido como
\begin{equation}\label{central}
    \mu_n = \sum_{i=0}^{L-1}{(r_i-m)^np(r_i)}
\end{equation}
Onde \(m\) é dado por
\begin{equation}\label{mean}
    m = \sum_{i=0}^{L-1}{r_ip(r_i)}
\end{equation}
\(m\) é uma medida da média de intensidade.
A variância (ou desvio padrão, \(\sigma\)), dado por
\begin{equation}\label{vari}
    \sigma^2 = \mu_2 = \sum_{i=0}^{L-1}(r_i-m)^2p(r_i)
\end{equation}
é uma medida de contraste.

As grandezas definidas acimas são \textit{globais}. 
São úteis para ajustes grossos na intensidade e contraste.
Grandezas \textit{locais} análogas a estas são mais poderosas, no que se utiliza de características da vizinhança de cada pixel.
O valor médio dos pixels em uma vizinhança \(S_{xy}\) é dado pela expressão
\begin{equation}\label{locmed}
    m_{S_{xy}} = \sum_{i=0}^{L-1}{r_ip_{S_{xy}}(r_i)}
\end{equation}
Sendo \(p_{s_{xy}}\) o histograma dos pixels na vizinhança \(S_{xy}\). 
A variância local, por sua vez, é dada por
\begin{equation}\label{locvar}
    \sigma^2_{S_{xy}} = \sum_{i=0}^{L-1}{(r_i-m_{S_{xy}})^2p_{S_{xy}}(r_i)}
\end{equation}

Analogamente as grandezas globais, a equação \ref{locmed} representa a intensidade média da região, enquanto aquela da equação \ref{locvar} representa o contraste.
É possível usar essas grandezas no aprimoramento de imagens.
Se uma imagem, em certas regiões muito escuras, contém detalhes quase imperceptíveis, aplica-se um aumento do contraste nessas regiões (e não nas demais), pelo critérios
de:
\begin{itemize}
    \item{Média de intensidade da região em comparação com a média global}
    \item{Variância local em comparação com a variância global}
\end{itemize}
Desta forma, somente os pixels que se enquadram nas duas características serão submetidos a uma função de intensidade.
Nesse caso, poderia ser simplesmente a multiplicação por uma constante \(C\).
Os pixels levemente mais claros se tornariam claros o suficiente para serem facilmente discernidos.
\subsection{Introdução a filtros espaciais}
A filtragem é um termo que vem do processamento no domínio da frequência, mas no domínio do espaço, significa uma transformação em cada pixel definida por uma função do valor do pixel e seus vizinhos.
Um filtro linear é definido pelo seu núcleo \(w\),
uma matriz cuja ordem define a vizinhança a ser utilizada no cálculo.
O pixel resultante nada mais é do que a soma dos produtos de cada elemento da matriz (núcleo) com o pixel correspondente da vizinhança. Algebricamente:
\begin{equation}\label{linfilt}
    g(x,y) = \sum_{s=-a}^a\sum_{t=-b}^b{w(s,t)f(x+s,y+t)}
\end{equation}
A função \texttt{fspecial} do Octave gera automaticamente alguns núcleos habitualmente utilizados. Com o núcleo em mãos, basta aplicar a função \texttt{imfilter}, usando como argumento a imagem e o núcleo, que a equação \ref{linfilt} é calculada para todos os pixels da imagem.

Uma vez que núcleos são matrizes, pela álgebra linear é possível demonstrar que, se uma matriz é decomposta em produtos de outras matrizes menores, pode se aplicar o filtro em questão em múltiplas etapas (uma para cada matriz).
Isso reduz o número de operações realizadas, e o resultado final é o mesmo.

Em geral, há 3 maneiras de definir o núcleo:
\begin{enumerate}
    \item{Baseado em propriedades matemáticas (por exemplo, média da vizinhança)}
    \item{Usar uma função de duas variáveis, cuja superfície tem uma propriedade de interesse}
    \item{Projetar um núcleo que tem uma frequência de resposta específica}
\end{enumerate}

\subsection{Filtros passa baixa (suavização)}
Filtros de suavização são usados para reduzir transições bruscas em intensidade. 
Como ruído consiste tipicamente destas, uma aplicação desses filtros é reduzir o ruído. Há diversas outras utilidades.
O efeito geral desses filtros é borrar a imagem, sendo que a intensidade desse efeito depende das dimensões do núcleo e o valor de seus elementos. 
Através de filtros passa baixa é possível gerar outros tipos (passa alta, passa faixa, rejeita faixa).
\subsubsection{Filtros de caixa}\label{boxsec}
Esses são os mais simples.
O núcleo tem todos seus elementos iguais a \(1 \div mn\), sendo \(m\) e \(n\) as dimensões do núcleo.
Quanto maior m e n, mais borrada a imagem fica, como ilustrado no exemplo a seguir.
O código abaixo, a partir da imagem da figura \ref{t1}, aplica um filtro de caixa de tamanho \(3 \times 3\), \(10 \times 10\) e \(20 \times 20\).
\lstinputlisting[language=octave]{box.m}
Os respectivos resultados estão nas figuras \ref{3x3}, \ref{10x10} e \ref{20x20}.

\begin{figure}[h]
    \centering
    \begin{minipage}{0.48\textwidth}
        \centering
  \includegraphics[width=.7\linewidth]{trinker.jpg}
    \caption{trinker.jpg}\label{t1}
    \end{minipage}
    \begin{minipage}{0.48\textwidth}
        \centering
  \includegraphics[width=.7\linewidth]{3x3.jpg}
        \caption{Caixa 3x3}\label{3x3}
    \end{minipage}
\end{figure}
\begin{figure}[h!]
    \centering
    \begin{minipage}{0.48\textwidth}
        \centering
  \includegraphics[width=.7\linewidth]{10x10.jpg}
    \caption{Caixa 10x10}\label{10x10}
    \end{minipage}
    \begin{minipage}{0.48\textwidth}
        \centering
  \includegraphics[width=.7\linewidth]{20x20.jpg}
        \caption{Caixa 20x20}\label{20x20}
    \end{minipage}
\end{figure}
\newpage
\subsubsection{Núcleos Gaussianos}
A técnica de filtros de caixa (\ref{boxsec}) é útil, mas falha em algumas circunstâncias. 
Por exemplo, ela favorece o desfoque em direções perpendiculares, o que em imagens com componentes geométricos, produz resultados indesejados.
Em situações como esta, são escolhidos filtros de núcleo simétricos circularmente, dos quais o único tipo que é separável é o Gaussiano, definido assim:
\begin{equation}
    w(s,t) = G(s,t) = Ke^{-\frac{s^2+t^2}{2\sigma^2}}
\end{equation}
que, definido \(r\) como \([s^2+t^2]^{\frac{1}{2}}\), pode ser reescrita como 
\begin{equation}
    w(r) = G(r) = Ke^{-\frac{r^2}{2\sigma^2}}
\end{equation}
Essa forma ilustra a propriedade circular dos filtros Gaussianos, ou seja, o fato de que o valor de cada elemento do núcleo depende apenas da distância ao centro (e das constantes \(K\), e \(\sigma^2\)).
Valores de \(r\) maiores que \(3\sigma\) são pequenos o bastante para serem ignorados.
Segue o exemplo em Octave, sendo \(K = 1\) e \(\sigma = 5\) em todos os casos, utilizando a imagem da figura \ref{coringa}:
Os respectivos resultados estão nas figuras \ref{g1}, \ref{g2} e \ref{g3}.
Embora entre os filtros de tamanho \(r = 5\) e \(r = 25\) a diferença seja grande, ela é quase imperceptível entre \(r = 25\) e \(r = 45\).
\lstinputlisting[language=octave]{gaussian.m}

\begin{figure}[h]
    \centering
    \begin{minipage}{0.48\textwidth}
        \centering
  \includegraphics[width=.7\linewidth]{coringa.jpg}
    \caption{coringa.jpg}\label{coringa}
    \end{minipage}
    \begin{minipage}{0.48\textwidth}
        \centering
  \includegraphics[width=.7\linewidth]{g1.jpg}
        \caption{\(r = 5\)}\label{g1}
    \end{minipage}
\end{figure}
\begin{figure}[h!]
    \centering
    \begin{minipage}{0.48\textwidth}
        \centering
  \includegraphics[width=.7\linewidth]{g2.jpg}
        \caption{\(r = 25\)}\label{g2}
    \end{minipage}
    \begin{minipage}{0.48\textwidth}
        \centering
  \includegraphics[width=.7\linewidth]{g3.jpg}
        \caption{\(r = 45\)}\label{g3}
    \end{minipage}
\end{figure}

\newpage
\subsubsection{Filtros não Lineares}
Esse tipo de filtros tem seus núcleos definidos pelos pixels contidos na região do filtro, o que caracteriza sua não linearidade.
O exemplo mais conhecido é o filtro mediano, que usa a  mediana da vizinhança. 
Ele é muito usado em redução de ruído.
Segue o exemplo em Octave, usando a imagem com ruído da figura \ref{noised}.
Note que a maior parte do ruído é removido (figura \ref{median})
\lstinputlisting[language=octave]{median.m}
\begin{figure}[h]
    \centering
    \begin{minipage}{0.48\textwidth}
        \centering
  \includegraphics[width=.9\linewidth]{noised.jpg}
    \caption{Imagem com ruído}\label{noised}
    \end{minipage}
    \begin{minipage}{0.48\textwidth}
        \centering
  \includegraphics[width=.9\linewidth]{median.jpg}
        \caption{Imagem após filtro mediano}\label{median}
    \end{minipage}
\end{figure}

\newpage

\subsection{Filtros Passa Alta}
Filtros desse tipo destacam as transições em intensidade (sharpening).
Eles são definidos por técnicas de ''diferenciação digital``.
\subsubsection{Filtro Laplaciano}
Discretamente, o \textit{Laplaciano} de duas variáveis é
\begin{equation}
    \nabla{f(x,y)} = f(x+1, y) + f(x-1, y) + f(x, y+1) + f(x, y-1) - 4f(x,y)
\end{equation}
Ele é usado portanto, da seguinte forma como filtro:
\begin{equation}
    g(x,y) = f(x,y) + c[\nabla{f(x,y)}] 
\end{equation}

Como o exemplo a seguir mostra, o filtro laplaciano destaca as transições de intensidade (figura \ref{lapla}), 
mas muita informação da imagem original é perdida.
Pode-se somar então a imagem original com este resultado, obtendo de volta as informações perdidas, mas mantendo o destaque de transições (figura \ref{added}).
\newpage
\lstinputlisting[language=octave]{laplacian.m}
\begin{figure}[h]
    \begin{center}
        \includegraphics[width=.7\linewidth]{landscape.jpg}
    \end{center}
    \caption{Original}\label{t2}
\end{figure}
\begin{figure}[h]
    \begin{center}
        \includegraphics[width=.7\linewidth]{laplacian.jpg}
    \end{center}
    \caption{Laplaciano}\label{lapla}
\end{figure}
\begin{figure}[h]
    \begin{center}
        \includegraphics[width=.7\linewidth]{added.jpg}
    \end{center}
    \caption{Soma}\label{added}
\end{figure}

\newpage
\subsubsection{Unsharp Masking}
Outro método de destacar transições de intensidade consiste em subtrair a imagem desfocada (usando um filtro passa baixa) da imagem original. 
Então pode-se adicionar este resultado com a imagem original.

\lstinputlisting[language=octave]{unsharp.m}
\begin{figure}[h!]
    \begin{center}
        \includegraphics[width=.7\linewidth]{landscape.jpg}
    \end{center}
    \caption{Original}\label{t2}
\end{figure}
\begin{figure}[h!]
    \begin{center}
        \includegraphics[width=.7\linewidth]{mask.jpg}
    \end{center}
    \caption{Mask}\label{lapla}
\end{figure}
\begin{figure}[h!]
    \begin{center}
        \includegraphics[width=.7\linewidth]{unsharp.jpg}
    \end{center}
    \caption{Original + Mask}\label{added}
\end{figure}
 
\subsubsection{Gradiente}
Por fim, ainda é possível usar o Gradiente como técnica de \textit{sharpening}.

\subsection{Filtros passa alta, passa faixa, e rejeita faixa a partir de filtros passa baixa}
Como mencionado anteriormente, filtros passa baixa \(lp\) podem ser usados para gerar outros filtros: (\(\delta\) é a imagem)

\begin{center}
\begin{tabular}{ |l|l| } 
 \hline
    Lowpass    & \( lp(x, y)                                                             \) \\
    Highpass   & \( hp(x, y) = \delta(x, y) - lp(x, y)                                   \) \\
    Bandreject & \( br(x, y) = lp_1(x, y) + [ \delta(x, y) - lp_2 (x, y) ]              \) \\
    Bandpass   & \( bp(x, y) = \delta(x, y) - \big[lp 1 (x, y) + [ \delta(x , y) - lp_2(x, y )]\big] \) \\
 \hline
\end{tabular}
\end{center}

\printbibliography
\end{document}
